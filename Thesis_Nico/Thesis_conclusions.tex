\chapter[Conclusions]{Conclusions\label{ch:concl}}

\begin{abstract}
The project aims to develop and characterize an ultracold ion source to be integrated in focused ion beam machines. The UCIS has been experimentally characterized in the experiments presented in chapter 3, chapter 4 and in the measurements performed with the same setup by Reijnders \textit{et al} \cite{Reijnders_PRL_09}.
Moreover, simulations of the ABLIS, another laser-cooled ions source, have been performed and are shown in chapter \ref{ch:newsource}. The results show that with the ABLIS it is possible to achieve a high brightness and a low energy spread at the same time, a result which is not possible to achieve with the UCIS. 
\end{abstract}

\clearpage

\section{UCIS}
In chapter 3, we measured the source temperature of the UCIS to be $(3 \pm 2)$ mK, with the use of time-dependent electric fields, which is a new method of focusing $^{85}$Rb$^+$ bunches. The expected source temperature of $T_0' = 390~\mu$K is consistent with this result given the error bounds. The reduced emittance is calculated to be $\epsilon_r = 1.4 \times 10^{-8}$ m rad $\sqrt{eV}$ for an effective source temperature of 3 mK and an initial source size $\sigma_{x_i}~=~38~\mu$m. The only other emittance measurements of an ultracold ion source were performed by Hanssen \textit{et al.} \cite{Hanssen_NL_08}. Their measured reduced emittance is a factor 23 smaller due to the combination of a lower source temperature (Chromium has a slightly smaller Doppler temperature \cite{Metcalf_Book_99}) and a smaller initial source size: $\sigma_{x_i} = 5$ $\mu$m. In fact, due to the large experimental uncertainty, the measured source temperatures presented in chapter 3 and in reference \cite{Hanssen_NL_08} agree within two standard deviations. This confirms that the effective temperature of the UCIS is indeed close to that of the laser-cooled atoms, which is an essential ingredient to achieve high brightness with the UCIS.

In chapter 4, we described a semi-analytic model that allows one to calculate the current that can be extracted from an ultracold ion source under pulsed operation. An important feature of the model is that it takes into account the radiation pressure exerted by the excitation laser beam. This turns out to have an important influence on the magnitude and temporal behavior of the current because it pushes atoms through the ionization region. The model was used to predict the optimum settings of ionization and loading times to achieve maximum cycle-averaged current. Measurements of extracted currents show general agreement with the predictions in terms of temporal behavior, magnitude of the current, dependence on the position of the ionization volume, and loading and ionization times for which optimum current extraction is achieved. The maximum cycle-average current we obtained was $\overline{I}=(13 \pm 1)$~pA. This corresponds to a current density $J= 4.9$ $10^{-3}$ A/m$^2$. If we assume that the transverse velocity spread of the ions is simply given by the temperature of the trapped atoms as found by Hanssen \emph{et al.} \cite{Hanssen_NL_08}, the transverse reduced brightness of the source is $B_r = (8 \pm 1) \times 10^4$~A/m$^2$~sr~eV. This not far from the brightness of $3\times 10^5$~A/m$^2$~sr~eV we predicted as ultimately achievable in \cite{Geer_JoAP_07}, but it is obtained for an atom number density that is 40 times lower and a smaller ionization probability per atom than used in that estimate. Under those circumstances, we calculate an \emph{rms} energy spread of the ion beam $\sigma_U= 0.9$ eV, well below that of the LMIS. We have therefore simulated extraction of the ions with the {\sc GPT} particle tracking code \cite{gpt} under conditions similar to the experiment. The 50\% reduced brightness \cite{Geer_JoAP_07} of the ion beam is calculated at a distance of 0.1~m behind the ionization region. Stochastic heating already decreases brightness starting at currents as low as 1~pA, where the reduced brightness is $2 \times 10^3$~A/m$^2$~sr~eV. With a higher energy spread of 4.5~eV, similar to the one of the LMIS, the disorder induced heating is ignorable up to 10 pA extracted current, with a reduced brightness value of $10^4$~A/m$^2$~sr~eV. Operating the source in a DC mode and with the use of a more powerfull ionization laser beam, can lead to a higher brightness.

\section{ABLIS}
In chapter 5, we explore whether the ABLIS, a difference concept of ion source, could improve the performance of the UCIS. We demonstrate with simulations the high potential of this source in terms of brightness and energy spread. The strength of the source is in its compactness and high atomic flux effusing from a Knudsen cell. The simulations are performed with realistic rubidium. The $^{87}$Rb isotope is found to reach higher atomic beam reduced brightnesses due to its lower nuclear spin with respect to the $^{85}$Rb isotope. The upper limit of the reduced peak brightness equals $10^8$~A/m$^2$~sr~eV in the best case scenario. Taking into account disorder-induced heating, reduces the ionic beam reduced peak brightness by 2-3 orders of magnitude depending on the aperture size $A_f$. But in certain conditions, the so-called pencil-beam regime is reached and the disorder-induced heating does not deteriorate the brightness anymore. In this regime, a reduced brightness of $2 \times 10^7$~A/m$^2$~sr~eV with an ion current of 20 pA and a 50\% energy spread of 0.7 eV can be achieved. This is obtained with a $^{87}$Rb$^+$ beam originating from a purified source. Another simulation, performed at higher extraction field, resulted in a beam of 55 pA with a reduced brightness of $2 \times 10^7$~A/m$^2$~sr~eV and a 50\% energy spread of 3.3~eV. Those results are very promising. Potentially the ABLIS is able to achieve a higher brightness and simultaneosly a lower energy spread than the LMIS. Finally, it is also possible to extract very high current (1~nA), making the ABLIS ideal also for milling applications with FIBs. Further investigation through simulations and experimentation may be required to realize such a source.

\section{Source comparison}
Table \ref{tab:sources} summarizes the most important properties of the sources discussed in this thesis: the liquid-metal ion source (LMIS), the gas field ionization source (GFIS), the ultracold ion source (UCIS) and the atomic beam laser-cooled ion source (ABLIS). The properties of the LMIS and the GFIS are taken from the literature (see table for details of the references). The properties of the UCIS and the ABLIS have been measured experimentally, except for the spot diameter which is calculated with equation (\ref{eq:minspot}), assuming that all the sources are chromatic-aberration-limited. The properties are valid for a probe current of 1~pA, normally used for microscopy applications. 

\begin{table}[h!]
\caption{\label{tab:sources} Summary of the most important properties of the ion sources used for FIB applications for a current of 1~pA. The spot size is calculated with equation (\ref{eq:minspot}), assuming that all the sources are chromatic-aberration-limited. The properties for the UCIS and the ABLIS are taken respectively from chapters \ref{ch:currentmea} and \ref{ch:newsource}. In more detail, the source size is calculated with equation (\ref{eq:f_diameter}), the brightness of the UCIS is extrapolated from figure \ref{fig:GPTresult} at 1~pA current and the brightness of the ABLIS from figure \ref{fig:GPT_result}. It should be noted that the performance of the UCIS was limited by the power of the ionization laser beam, as explained at the end of chapter \ref{ch:currentmea}, and by the fact that it operates in a pulse mode. DC operation and a higher laser power will result in a higher average brightness.}
\begin{center}
  \begin{tabular}{ | c | c | c | c | c | }
  \hline
  Source property	(unit) & LMIS~\cite{Hagen_J_08, Bell_JVSTB_88}  & GFIS~\cite{Ward_JVSTB_06} & UCIS  	& ABLIS				\\ \hline
  Used element	& $^{69}$Ga		& $^4$He 		& $^{85}$Rb 		& $^{87}$Rb					\\ 
  Source diameter $d_s$ (nm)	& 50	& $\sim 0.6$ 		& $76 \times 10^3$ 		&		$332$	\\ 
  %Current $I$ (pA)		&  1		& 	1	& 20		& 1  					\\
  Energy spread (eV) 		& 4.5 (FWHM)		& < 0.1 		& 0.9 (\textit{rms})			& 0.7 (50 \%) 	\\
  Brightness $B_r$ (A / m$^2$ sr eV)		& $10^6$		& $5 \times 10^8$ 		& $2 \times 10^3$		& $3 \times 10^7$		 \\
  Spot diameter $d_{50}$ (nm) & 3 & 0.3 & 9 & 0.5 \\ \hline
  \end{tabular}
\end{center}
\end{table}

The estimation of the attainable spot diameter $d_{50}$ implies that using the ABLIS in a FIB apparatus can lead to a spot diameter below the nanometer, better than the LMIS and almost at the same level of a GFIS. The ABLIS seems to fulfill the requirements of the best of both worlds: on the one hand, it is able to improve the performance of the LMIS in both terms of reduced brightness and energy spread and, on the other hand, it allows the use of many more atomic species than the LMIS (gallium and a few more) and GFIS (helium and neon). %This source is not only useful for microscopy, but for milling applications as well, since it allows the use of heavier elements than neon.
Since the ABLIS allows the use of both light elements, e.g. the alkali lithium, and heavy elements, e.g. cesium, it is relevant both to ion microscopy and ion milling applications.

\section{Future prospects}
At the Eindhoven University of Technology, a new project has been granted. This project aims at the experimental realization of the ABLIS, based on the predictions from the simulations shown in chapter 6 of this dissertation. The project is part of a collaboration with FEI Company \cite{FEI}. After building the source, the first objective will be to show that the predicted high brightness and low energy spread can be realized experimentally. Subsequently the source will be mounted on an existing focusing column obtained from FEI Company, which should lead to unprecedented ion imaging and sputtering capabilities.

\clearpage

\bibliographystyle{unsrt}
\begin{thebibliography}{10}

\bibitem{Reijnders_PRL_09}
M.~P. Reijnders, P.~A. van Kruisbergen, G.~Taban, S.~B. van~der Geer, P.~H.~A.
  Mutsaers, E.~J.~D. Vredenbregt, and O.~J. Luiten.
\newblock {L}ow-{E}nergy-{S}pread {I}on {B}unches from a {T}rapped {A}tomic
  {G}as.
\newblock {\em Phys. Rev. Lett.}, 102(3):034802, Jan 2009.

\bibitem{Hanssen_NL_08}
J.~L. Hanssen, S.~B. Hill, J.~Orloff, and J.~J. McClelland.
\newblock {M}agneto-{O}ptical-{T}rap-{B}ased, {H}igh {B}rightness {I}on
  {S}ource for {U}se as a {N}anoscale {P}robe.
\newblock {\em Nano Lett.}, 8(9):2844--2850, September 2008.

\bibitem{Metcalf_Book_99}
H.J. Metcalf and P.~van~der Straten.
\newblock {\em {L}aser {C}ooling and {T}rapping}.
\newblock Springer, 1999.

\bibitem{gpt}
http://www.pulsar.nl/gpt.

\bibitem{Geer_JoAP_07}
S.~B. van~der Geer, M.~P. Reijnders, M.~J. de~Loos, E.~J.~D. Vredenbregt,
  P.~H.~A. Mutsaers, and O.~J. Luiten.
\newblock {S}imulated performance of an ultracold ion source.
\newblock {\em J. Appl. Phys.}, 102(9):094312, 2007.

\bibitem{Hagen_J_08}
C.~W. Hagen, E.~Fokkema, and P.~Kruit.
\newblock {B}rightness measurements of a gallium liquid metal ion source.
\newblock {\em J Vac Sci Technol B}, 26(6):2091--2096, 2008.

\bibitem{Ward_JVSTB_06}
B.W. Ward, J.A. Notte, and N.P. Economou.
\newblock {H}elium ion microscope: {A} new tool for nanoscale microscopy and
  metrology.
\newblock {\em J. Vac. Sci. Technol. B}, 24/6:2871--2874, 2006.

\bibitem{Bell_JVSTB_88}
A.E. Bell, K.~Rao, G.A. Schwind, and L.W. Swanson.
\newblock A low-current liquid metal ion source.
\newblock {\em J. Vac. Sci. Technol. B}, 6:927--931, 1988.

\bibitem{FEI}
http://www.fei.com.

\end{thebibliography}

%\end{document}