\chapter*{Summary}
{\bf UltraCold Ion Beam Source.} Focused ion beam (FIB) machines are largely used in the semiconductor industry mainly for imaging, milling and deposition of material, at the nanometer scale. To keep up with the trend of making smaller and smaller components, FIBs need to be continuously improved. A FIB is mainly made of an ion source and a focusing column. In this thesis, two concepts of ion sources are presented, modeled and discussed.

The ultra-cold ion source (UCIS) is based on creating very cold ion beams ($T < 1$~mK) by near-threshold photo-ionization of a laser-cooled and trapped atomic gas. In our experiment, $^{85}$Rb atoms are confined in a magneto optical trap built directly inside an accelerator structure. A spherical portion of the atom cloud is ionized with a 2-step ionization process and the bunched ions are accelerated by either a DC or a pulsed electric field. The liquid-metal ion source (LMIS) is the current state-of-art for focused ion beam technology, having a reduced brightness of $10^6$~A/m$^2$~sr~eV. The reduced brightness characterizes the source and is proportional to the ion current and inversely proportional to the beam emittance (closely related with the source temperature).  A high brightness is of importance as much as a low energy spread of the ion beam, since the latter is related with the focus ability of the beam. We have already shown that the UCIS can provide much lower energy spread (20 meV) than LMIS ($\approx$~4.5 eV), see Reijnders \textit{et al.} (PRL, 2009), but not yet in combination with a high brightness as well. The energy spread is the limiting factor for the final achievable spot size in a FIB based on a LMIS. Chapter 3 and 4 present measurements of the source temperature and of the extracted current to complete the characterization of the UCIS. 

The ultra low temperature of the source permits collimated bunches to be created at low kinetic energy (down to few eV), which allows using time-dependent fields for accelerating and focusing. With this lens, a transversal ion temperature of $(3 \pm 2)$~mK has been measured, as shown in chapter 3. This temperature is an upper limit since it is ten times higher than the expected value. The novel technique of focusing with time-dependent electric field is successfully applied to an ultra cold ion beam. In chapter 4, a dynamic model of the source describing its properties under pulsed operation has been developed and experiments validated it. A maximum average ion current of $(13 \pm 1)$~pA has been extracted from the UCIS and this is enough for imaging (1-100~pA are used). Those results correspond to a source reduced brightness of $(8 \pm 1) \times 10^4$~A/m$^2$~sr~eV with 0.9~eV energy spread. This reduced brightness of the UCIS is 12.5 times lower than the one of a LMIS. An interesting "pushing effect", due to the laser used to excite the atoms in the cloud, is described as well. Particle tracking simulations show that disorder induced heating has an important effect on the brightness, lowering its value by 1-2 orders of magnitude depending on the extracted current.

Calculations and simulations of a different ion source, the atomic beam laser-cooled ion source (ABLIS), are presented in chapter 6. The follow-up project is based on a similar idea as an UCIS but aims for production of ion beams from a much higher initial atomic flux. The atoms originate from a Knudsen cell and, after being cooled and compressed transversally with laser cooling and trapping techniques, are photo-ionized and accelerated. The source size is reduced as well with respect to the UCIS. Realistic particle tracking simulations with $^{87}$Rb show that disorder induced heating is still very important, but in certain conditions, the so-called ``pencil-beam'' regime is reached and the disorder-induced heating does not deteriorate the brightness anymore. Depending on the extracted current, it is possible to reach a reduced brightness of $2 \times 10^7$~A/m$^2$~sr~eV with an ion current of 22 pA (corresponding to a final aperture diameter $d_f=1.6~\mu$m) and an energy spread of 0.7 eV can be achieved. This source is likely capable of provide a high brightness and a low energy spread at the same time. The calculations are done assuming a purified $^{87}$Rb source. The estimation of the attainable spot diameter implies that using the ABLIS in a FIB apparatus can lead to a spot diameter below the nanometer. The ABLIS seems to fulfill the requirements of the best of both worlds: on the one hand, it is able to improve the performance of the LMIS in both terms of reduced brightness and energy spread and, on the other hand, it allows the use of many more atomic species than the existing FIB sources. Since the ABLIS allows the use of both light elements, e.g. the alkali lithium, and heavy elements, e.g. cesium, it is relevant both to ion microscopy and ion milling applications.
 
