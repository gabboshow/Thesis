\documentclass[12pt]{iopart}
%\documentclass[12pt,a4paper]{paper}

\usepackage{graphicx}% Include figure files
\usepackage{dcolumn}% Align table columns on decimal point1
\usepackage{bm}% bold math
\usepackage[caption=false]{subfig}

\begin{document}
\title{Introduction}

%\chapter[Introduction]{Introduction\label{ch:intro}}

\begin{abstract}
The work in this thesis covers the research on two kinds of laser-cooled ion sources for focused ion beam (FIB) applications. An ion source is an apparatus capable of providing ions preferably with a high current density. This section summarizes the applications of FIBs and gives an overview over the existing ion sources. The reduced brightness and the emittance are useful figures of merit used to characterize an ion source and a definition is here given. Those definitions are used through the chapters to compare the performances of the sources. Finally, an outline of this thesis is given at the end of this chapter.
\end{abstract}

\section{Overview over focused ion beam applications}
Miniaturization is a central theme in modern fabrication technology and many of the components used in modern products are becoming smaller and smaller \cite{Moore_E_65}. Nanotechnology tools have to keep up with this trend by improving minimum achievable resolution. FIB systems have in common high-brightness ion sources capable of producing high current density beams \cite{Orloff_RoSI_93, Orloff_Book_03}. Those machines can create structures with few nanometer precision, remove or add material through ion bombardment or beam-induced chemistry. Imaging is also possible, although damages are dominant at the smallest scales. Figure \ref{fig:applications}, shows three examples of the most common applications of FIBs. 
\begin{figure}[tbh!]
    \centering
    \subfloat[Imaging]{
        \label{a1}
        \includegraphics[width=0.3\linewidth]{appl1}}
    \hspace{.1in}
    \subfloat[Milling]{
        \label{a2}
        \includegraphics[width=0.3\linewidth]{appl2}}
    \hspace{.1in}
    \subfloat[Deposition]{
        \label{a3}
        \includegraphics[width=0.3\linewidth]{appl3}}
    \caption{Three examples of application of FIBs. The figure is taken from \cite{Reyntjens2001}.}
    \label{fig:applications}
\end{figure}

A focused ion beam microscope operates along the same principle as the scanning electron microscope (SEM), in that a beam of charged particles is scanned across a specimen, and the resultant signals at each raster position are plotted to form an image. The signal comes from a multichannel plate detector (MCP), see figure \ref{a1}. However, in FIBs the charged particle beam consists not of electrons, but rather of ions, which are typically positively charged. Because of the short De Broglie wavelength of the ions, FIBs have achieved spatial resolution rivaling that of the standard scanning electron microscope. Today's state-of-the-art FIBs are capable of imaging with a few nm spatial resolution, using secondary electrons or ions, i.e. secondary ion mass spectrometry (SIMS). Figure \ref{fig:FIBschem} shows a typical schematic of a FIB ion column.
\begin{figure}[tbh!]
    \centering
        \includegraphics[width=0.5\linewidth]{FIBschem}
    \caption{Schematic drawing of a FIB column \cite{Reyntjens2001}.}
    \label{fig:FIBschem}
\end{figure}
The ion are produced by the source on the top of the column. After a first refinement through the spray aperture, the ion beam is condensed in the first electrostatic lens and the spherical aberration is reduced with an octopole. The variable aperture is used to skim the beam to the desirable current. Blanking of the beam is accomplished by the blanking
deflector and aperture, while the lower octopole is used for raster scanning the beam over the sample in a user-defined pattern. In the second electrostatic lens, the beam is focused to a fine spot. The multichannel plate (MCP) is used to detect secondary particles for imaging.

The use of FIBs as precision sectioning tools down to sub-micron scale, creating a precise cut or cross-section from a few tens of nm with positional placement accuracy on the order of 20~nm is well exploited in the industry, see figure \ref{a2}. FIB has recently become a popular tool in making high-quality microdevices or high-precision microstructures \cite{Tseng2004}. Also, their ability to deposit metals and insulators on a micron scale is of large importance, introducing several different gases into the vacuum chamber to either deposit a range of materials up to tens of micrometers thickness over areas ranging from the sub-micrometers to tens of thousands of square micrometers or selectively etch one material rapidly, see figure \ref{a3}. 

Moreover, FIB have been adopted into the product cycle of semiconductor manufacturers in the late 1980s. In 1986, the number of FIB fabrication systems totaled about 35; in 1998, the number has increased 15-25 times \cite{Phaneuf1999}. FIBs are highly exploited in device modification, mask repair, process control and failure analysis \cite{Glanville1989, Ward1988, Stewart1995}. 
Even more, the preparation of specimens for transmission electron microscopy (TEM) is an important application \cite{Walker1995}.


\section{Overview of the existing ion sources for FIBs}
In the past few years, the interest on focused ion beams has seen another significant increase also because of the development of few distinct ion sources, which can expand the capabilities of these instruments. FIB system based on high-brightness \cite{Hagen_J_08} gallium liquid-metal ion sources (LMIS) became commercially available in the late 1980s. The LMIS \cite{Orloff_Book_03} is still currently the most used source in commercial instruments. The current is extracted from an extremely sharp tip (of the diameter of several nanometers) by applying high electric fields. Gallium is the most used element in the LMIS and this is also an important limitation for these sources. Expanding the choice of atomic species beyond gallium opens up FIBs to new applications on the market. Also, the resolution of LMIS is chromatic aberration limited to a few nanometers due to the energy spread of the source. The resolution of the LMIS will not be enough in the near future. The gas field ionization source (GFIS) \cite{Ward_JVSTB_06} replaces the heavy Ga$^+$ ions with much lighter He$^+$ (Ne$^+$) ions, enabling a very high resolution imaging. Long lifetime (this only applies to the Helium source, for Neon the lifetime is up to ten hours \cite{Notte2010}) and high brightness operation are its most important features. The resolution obtained with an ion microscope based on GFIS is probably unbeatable, but the milling capacities are quite poor. Especially due to the high penetration in the sample, the sputter yield is very low, almost neglecting the advantage of having a really high brightness \cite{Tan2010}. In order to improve the milling application, an inductive-coupled plasma source is being developed \cite{Smith2006}. This source uses Ar$^+$ and Xe$^+$ ion beams that can remove material almost contamination-free.

Laser-cooled ion sources as the ultracold ion source (UCIS) \cite{Claessens2005} and the magneto-optical trap ion source (MOTIS) \cite{Hanssen_PRA_06}, promises to expand the choice of ions available for FIB applications and aim to improve the milling and circuit editing capability for FIBs. Those sources can in principle be configured to produce ions of over twenty different elements \cite{Steele_JVSTB_10}. Both of the sources use the principle of laser-cooling and trapping of neutral atoms \cite{Metcalf_Book_99}, which are photo-ionized and accelerated to form a ion beam or ion bunches. Those sources aim to achieve high brightness, comparable to the one of the LMIS, by a very low transversal temperature (a few hundreds of microkelvin) of the ions, rather than extracting the ions from a ``point source''. Having an extended source is advantageous for the energy spread because space charge density is reduced and the probe is less sensitive to vibrations and instabilities, as one typically experiences with point sources. Because of these characteristic of the source, sub-micrometer focus-ability has been demonstrated \cite{Steele_JVSTB_10}. The energy spread has been measured down to 20 meV \cite{Reijnders_PRL_09}. Time-dependent manipulation of ion bunches has been shown in references \cite{Reijnders_PRL_10, Reijnders_JAP_11} and is successfully used for the measurement presented in chapter 3.

%To avoid disorder-induced heating of the UCIS \cite{Geer_JAP_07} and the MOTIS \cite{Steele_JAP_11}, another laser-cooled source with a much higher particle flux is proposed in chapter 6. 

Finally, laser-cooled sources have been successful in the production of bright electron beams for ultrafast electron diffraction \cite{Taban_EPL_10, Mcculloch2011}. 

\section{Source characterization: useful figures of merit}
When characterizing an ion source, only two figures of merit are of most importance: the brightness and the longitudinal energy spread.

%The reduced (transversal) brightness $B_r$ \cite{Luiten2007} is the current density per unit of solid angle and beam energy and is a in Lorentz-invariant. In units of [A/m$^2$~sr~eV], the reduced brightness is defined as
%\begin{equation}
%	B_r = \frac{2 I}{4 \pi^2 m c^2 \epsilon_n^x \epsilon_n^y},
%	\label{eq:red_brightness_intro}
%\end{equation}
%where $I$ is the current and $\epsilon_n^x$ and $\epsilon_n^y$ are the normalized \textit{rms}-emittance of the beam in the $x$ and $y$ direction (in this thesis, the ions always propagate along the $z$-axis). In the previous equation, a factor $e$ (the elementary charge) is removed in order to express the unit of brightness in $eV$ instead of $V$. The normalized \textit{rms}-emittance $\epsilon_n^x$ is defined as
%\begin{equation}
%	\epsilon_n^x = \frac{1}{m c} \sqrt{\langle x^2 \rangle \langle p_x^2 \rangle - \langle x p_x \rangle^2} = \frac{1}{c} \sqrt{\langle x^2 \rangle \langle v_x^2 \rangle - \langle x v_x \rangle^2}.
%	\label{eq:norm_emittance_intro}
%\end{equation}
%where $c$ is the speed of light, $x$ is the position of a particle and the notation $\langle~\rangle$ implies an average over a set of particles. In the previous equation, the substitution of the non-relativistic momentum $p_x = m v_x$ is used (ions at energies up to 15~keV are non-relativistic). The normalized \textit{rms} emittance $\epsilon_y^n$ is defined similarly. Filling equation (\ref{eq:norm_emittance_intro}) in equation (\ref{eq:red_brightness_intro}), the reduced brightness can be rewritten as
The reduced (transversal) brightness $B_r$ \cite{Luiten2007} is the current density per unit of solid angle. In units of [A/m$^2$~sr~eV] is given by
\begin{equation}
	B_r = \frac{I}{4 \pi^2 \epsilon_r^x \epsilon_r^y},
	\label{eq:red_brightness2_intro}
\end{equation}
where the reduced emittance in the $x$-direction $\epsilon_r^x$ is
\begin{equation}
	\epsilon_r^x = \sqrt{\frac{m}{2}} \sqrt{\langle x^2 \rangle \langle v_x^2 \rangle - \langle x v_x \rangle^2}.
	\label{eq:red_emittance_intro}
\end{equation}
Equation (\ref{eq:red_emittance_intro}) can be derived from the usual definition of reduced emittance $\epsilon_r^x=\sigma_x \sigma_{x'} \sqrt{U}$ that can be found in literature. A similar equation holds for the quantity $\epsilon_r^y$. In this thesis, the ions always propagate along the $z$-axis.\\

The \textit{rms} energy spread $\sigma_U$ of an UCIS is an intrinsic property of the source. It is proportional to the accelerating field $E_0$ and the ionization laser \textit{rms} radius $\sigma_i$ as
\begin{equation}
	\sigma_U = e E_0 \sigma_i.
	\label{eq:en_spread}
\end{equation}

The attainable spot size $d$ in a FIB \cite{Kruit1997}, for a chromatic aberration limited source, is connected to the energy spread and the brightness as
\begin{equation}
	d=\left(\frac{I~C_C^2~\sigma_U^2}{B_r~V_p^3}\right)^{1/4},
	\label{eq:minspot}
\end{equation}
where $C_C$ is the chromatic aberration coefficient of the focusing system, $I$ is the ion beam current and $V_p$ is the voltage applied to accelerate the ions. Typically, $C_C=20$~mm and $V_p=30$~kV for FIB columns \cite{Geer_JAP_07}.

%\begin{table}[tbh!]
% \caption{Summary of the characteristics of the ion sources for FIBs.}
%\begin{tabular}{}
%
%    \label{tab:sources}
%                                          & \textbf{UCIS}                       & \textbf{LMIS}                         & \textbf{FES} \\
%                                          & Ultra Cold                          & Liquid Metal                          & Field Emission        \\
%                                          & Ion Source                          &       Ion Source                      &   Source \\
%
%    \textbf{Properties}\\
%                \hspace{10mm}Current $I$  & $0-100\rm~pA$                       & $0-10\rm~nA$                          & $0-100\rm~pA$\\
%                \\
%                \hspace{10mm}Ion species  & Alkali metals                       & Ga                                    & He\\
%                                          & (e.g. Li, Cs), \\
%                                          & Alkaline earth metals,\\
%                                          & Cr, Yb, \\
%                                          & Nobel gasses, ...\\
%
%    \textbf{Beam quality (1 pA)}  \\
%    \hspace{10mm}Brightness $B_r$         & $3 \times 10^5$                     & $1 \times 10^6$                       & $5 \times 10^8$\\
%                                          & \small{A/(m$^{2}$ sr V)}            & \small{A/(m$^{2}$ sr V)}              & \small{A/(m$^{2}$ sr V)}\\
%    \\
%    \hspace{10mm}Energy spread $\Delta U$ & $0.2~\rm eV$                        & $4.5~\rm eV$ \cite{LMIS-energyspread}  & $<1.0~\rm eV$ \cite{He-microscope2}\\
%    \\
%    \hspace{10mm} Source radius $R$       & $4.5\rm~\mu m$                      & $25~\rm nm$ \cite{LMIS-brightness1}    & $\sim 0.3~\rm nm$ \cite{He-microscope2}\\
%    \hspace{10mm} (effective)\\
%    \\
%    \hspace{10mm}Temperature $T$          & $150\rm~\mu K$                      &                                       & \\
%
%\end{tabular}
%\end{table}


\section{This thesis}
This thesis is mainly made up as a collection of three scientific papers, already published or submitted for publication (chapters 3, 4 and 5). In chapter 2, the experimental setup is presented, focusing on some details that were not covered in the publications. Chapter 3 and 4 are based on experimental measurements of the UCIS. Chapter 3 deals with the measurement of the source temperature. An upper limit of the emittance is there obtained and this chapter introduces a new technique used to focus an ion beam with the use of time-dependent electric fields. In chapter 4, a model of the source is presented and validated with measurements. Then, the current of the UCIS is measured and finally simulations of disorder-induced heating are presented. The general conclusion is that the UCIS can not increase the brightness and at the same time decrease the longitudinal energy spread beyond that of the LMIS: one of the two quantities needs to be sacrificed with respect to the other one. To overcome this problem, chapter 5 presents a new kind of source, also based on laser-cooling, and summarize a series of simulations where it is shown that even with taking into account the disorder induced heating, the source performs better than the LMIS. Finally, chapter 6 summarizes the main conclusions of the thesis and gives an outlook for the future of laser-cooled ion sources for FIBs.\\

\bibliographystyle{unsrt}
%\bibliography{biblio}

\begin{thebibliography}{10}

\bibitem{Moore_E_65}
G.~E.~Moore.
\newblock {C}ramming more components onto integrated circuits.
\newblock {\em Electronics}, 38/8, 1965.

\bibitem{Orloff_RoSI_93}
J.~Orloff.
\newblock {H}igh-resolution focused ion beams.
\newblock {\em Rev. Sci. Instrum.}, 64(5):1105--1130, 1993.

\bibitem{Orloff_Book_03}
J.~Orloff, M.~Utlaut, and Swanson L.
\newblock {\em {H}igh {R}esolution {F}ocused {I}on {B}eams: {FIB} and {I}ts
  {A}pplications}.
\newblock Kluwer Academic, 2003.

\bibitem{Reyntjens2001}
S.~Reyntjens and R.~Puers.
\newblock A review of focused ion beam applications in microsystem technology.
\newblock {\em J. Micromech. Microeng.}, 11:287--300, 2001.

\bibitem{Phaneuf1999}
M.~W. Phaneuf.
\newblock Applications of focused ion beam mcroscopy to material scence
  specimens.
\newblock {\em Micron}, 30:277--288, 1999.

\bibitem{Glanville1989}
J.~Glanville.
\newblock Focused ion beam technology for integrated circuit modification.
\newblock {\em Solid State Technol.}, 32:270, 1989.

\bibitem{Ward1988}
B.~W. Ward, N.~P. Economou, D.~C. Shaver, J.~E. Ivory, M.~L. Ward, and L.~A.
  Stern.
\newblock Microcircuit modification using focused ion beams.
\newblock In {\em Proc. SPIE}, page~92, 1988.

\bibitem{Stewart1995}
D.~K. Stewart, A.~F. Doyle, and Casey J.~D. Jr.
\newblock Focused ion beam deposition of new materials: dielectric films for
  device modification and mask repair, and ta films for x-ray mask repair.
\newblock In {\em Proc. SPIE}, page 276, 1995.

\bibitem{Walker1995}
J.~F. Walker, J.~C. Reiner, and C.~Solenthaler.
\newblock Focused ion beam sample preparation for tem.
\newblock In {\em Proc. Microsc. Semicond. Mater. Conf.}, page 629, 1995.

\bibitem{Tseng2004}
A.~A. Tseng.
\newblock Recent developments on micromilling using focused ion beam
  technology.
\newblock {\em J. Micromech. Microeng}, 14:R15--R34, 2004.

\bibitem{Hagen_J_08}
C.~W. Hagen, E.~Fokkema, and P.~Kruit.
\newblock {B}rightness measurements of a gallium liquid metal ion source.
\newblock {\em J Vac Sci Technol B}, 26(6):2091--2096, 2008.

\bibitem{Ward_JVSTB_06}
B.W. Ward, J.A. Notte, and N.P. Economou.
\newblock {H}elium ion microscope: {A} new tool for nanoscale microscopy and
  metrology.
\newblock {\em J. Vac. Sci. Technol. B}, 24/6:2871--2874, 2006.

\bibitem{Tan2010}
S.Tan, R. Livengood, D. Shima, J. Notte, and S. McVey.
\newblock 54th international conference on electron, ion and photon beam technology and nanofabrication / Focused Ion Beams.
\newblock In {\em Gas field ion source and liquid metal ion source charged particle material interaction study for semiconductor nanomachining applications}. J. Vac. Sci. Technol. B 28, C6F15, 2010.

\bibitem{Notte2010}
J. Notte, F. Rahman, S. McVey, S. Tan, and R. H. Livengood
\newblock Neon Gas Field Ion Source � Stability and Lifetime.
\newblock {\em Microsc. and Microanal.}, 16(2), 2010.

\bibitem{Smith2006}
N.~S. Smith, W.~P. Skoczylas, S.~M. Kellogg, D.~E. Kinion, P.~P. Tesch,
  O.~Sutherland, A.~Aanesland, and R.~W. Boswell.
\newblock High brightness inductively coupled plasma source for high current
  focused ion beam applications.
\newblock {\em J. Vac. Sci. Technol. B}, 24(6):2902--2906, 2006.

\bibitem{Claessens2005}
B.~J. Claessens, S.~B. van~der Geer, G.~Taban, E.~J.~D. Vredenbregt, and O.~J.
  Luiten.
\newblock {U}ltracold {E}lectron {S}ource.
\newblock {\em Phys. Rev. Lett.}, 95(16):164801, 2005.

\bibitem{Hanssen_PRA_06}
J.~L. Hanssen, J.~J. McClelland, E.~A. Dakin, and M.~Jacka.
\newblock {L}aser-cooled atoms as a focused ion-beam source.
\newblock {\em Phys. Rev. A: At. Mol. Opt. Phys.}, 74(6):063416, 2006.

\bibitem{Steele_JVSTB_10}
A.~V. Steele, B.~Knuffman, J.~J. McClelland, and J.~Orloff.
\newblock {F}ocused chromium ion beam.
\newblock {\em J Vac Sci Technol B}, 28(6):C6F1--C6F5, 2010.

\bibitem{Metcalf_Book_99}
H.J. Metcalf and P.~van~der Straten.
\newblock {\em {L}aser {C}ooling and {T}rapping}.
\newblock Springer, 1999.

\bibitem{Reijnders_PRL_09}
M.~P. Reijnders, P.~A. van Kruisbergen, G.~Taban, S.~B. van~der Geer, P.~H.~A.
  Mutsaers, E.~J.~D. Vredenbregt, and O.~J. Luiten.
\newblock {L}ow-{E}nergy-{S}pread {I}on {B}unches from a {T}rapped {A}tomic
  {G}as.
\newblock {\em Phys. Rev. Lett.}, 102(3):034802, 2009.

\bibitem{Reijnders_PRL_10}
M.~P. Reijnders, N.~Debernardi, S.~B. van~der Geer, P.~H.~A. Mutsaers, E.~J.~D.
  Vredenbregt, and O.~J. Luiten.
\newblock {P}hase-{S}pace {M}anipulation of {U}ltracold {I}on {B}unches with
  {T}ime-{D}ependent {F}ields.
\newblock {\em Phys. Rev. Lett.}, 105(3):034802, 2010.

\bibitem{Reijnders_JAP_11}
M.~P Reijnders, N.~Debernardi, S.B. van~der Geer, P.H.A. Mutsaers, E.J.D.
  Vredenbregt, and O.J. Luiten.
\newblock {T}ime-dependent manipulation of ultracold ion bunches.
\newblock {\em J. Appl. Phys.}, 109:033302, 2011.

\bibitem{Geer_JAP_07}
S.~B. van~der Geer, M.~P. Reijnders, M.~J. de~Loos, E.~J.~D. Vredenbregt,
  P.~H.~A. Mutsaers, and O.~J. Luiten.
\newblock {S}imulated performance of an ultracold ion source.
\newblock {\em J. Appl. Phys.}, 102(9):094312, 2007.

\bibitem{Steele_JAP_11}
A.~V. Steele, B.~Knuffman, and J.~J. McClelland.
\newblock {I}nter-ion coulomb interactions in a magneto-optical trap ion
  source.
\newblock {\em J. Appl. Phys.}, 109(10):104308, 2011.

\bibitem{Taban_EPL_10}
G.~Taban, M.~P. Reijnders, B.~Fleskens, S.~B. van~der Geer, O.~J. Luiten, and
  E.~J.~D. Vredenbregt.
\newblock {U}ltracold electron source for single-shot diffraction studies.
\newblock {\em EPL (Europhysics Letters)}, 91(4):46004, 2010.

\bibitem{Mcculloch2011}
A.~J. Mcculloch, D.~V. Sheludko, M.~Junker, S.~C. Bell, S.~D. Saliba, K.~A.
  Nugent, and R.~E. Scholten.
\newblock Arbitrary shaped high-coherence electron bunches from ultracold
  plasma.
\newblock {\em Nature Physics}, 7:785--788, 2011.

\bibitem{Luiten2007}
O.~J. Luiten, B.~J. Claessens, S.~B. van~der Geer, M.~P. Reijnders, G.~Taban,
  and E.~J.~D. Vredenbregt.
\newblock Proceedings of the 46th workshop of the infn eloisatron project.
\newblock In {\em The physics and applications of high brightness electron
  beams}. World Scientific Publishing Co. Pte. Ltd., 2007.
  
\bibitem{Kruit1997}
P.~Kruit, and H.~Jansen.
\newblock {\em Space charge and statistical Coulomb effects}.
\newblock in {\em Handbook of charged particle optics}.
\newblock Ed. Jon Orloff, CRC Press, New York, 1997.
  
\end{thebibliography}

\end{document}