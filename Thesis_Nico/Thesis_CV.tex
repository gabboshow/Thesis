\chapter*{Curriculum vitae}
\begin{wrapfigure}{l}{0.35\textwidth}
  \begin{center}
   \vspace{-20pt}
    \includegraphics[width=0.3\textwidth]{pics/nicola}
  \end{center}
   \vspace{-20pt}
\end{wrapfigure}
Nicola Debernardi was born on June 3 1981 in Biella (BI), Italy. He grew up around this small city in Piedmont in the Northwest of Italy, at the foot of the Alps. During his early scholar carrier he developed a strong interest for science and an affinity for scientific subjects. He therefore attended ``Liceo scientifico tecnologico ITIS'' in Biella, which is a scientific high school, propaedeutic for the University. He got the high school diploma in 2000.

After a mandatory year of military service, he followed Physics courses at the Univerity of Turin (Universita' degli Studi di Torino), from where in 2008 he obtained his Master degree in Biomedical Physics with Prof. Renzo Levi. On the same year, he got a master degree in Applied Physics from the Eindhoven University of Technology (located in the Netherlands) with Dr. Peter Mutsaers. The research for the master project was performed in Eindhoven in 2007 collaborating with Prof. Ger van der Vusse from University of Maastricht. From 2008, he lived to Eindhoven again where he developed the research described in this dissertation for about 4 years. The research was conducted under the supervision of Dr. Edgar Vredenbregt and Dr. Peter Mutsaers in the group Coherent and Quantum Technology (CQT) of Prof. Jom Luiten. 

During his PhD, he guided two students during their internship and two other ones during their master project. They strongly contributed to the research presented in chapters 3, 4 and 5 of this thesis. Beside that, in 2009 he received the price "STOOR Onderwijsprijs" from the bachelor students, for being best tutor for 2nd year students. 

Moreover, some of his results were presented at national symposia in Amsterdam, Veldhoven and Lunteren, and international conferences in Hannover (Germany), Gdansk (Poland), Gatlinburg (USA), Cairns (Australia) and Atlanta (USA).

After Physics, his biggest passion are photography and skateboarding. Especially the combination of the two activities, skateboard photography is evolving from a simple hobby to a side job. During the last years he managed to publish some of his pictures in a few international online magazines and he constantly keeps update his portfolio website at www.dobermaniprod.biz.